\subsection{Pflichtenheft (Auszug)}
\label{app:Pflichtenheft}

\subsubsection*{Zielbestimmung}
\begin{enumerate}
    \item Plattform
        \begin{enumerate}
            \item Die Anwendung wird in \Fachbegriff{C\# 10.0} umgesetzt.
            \item Das benutzte Framework ist \Fachbegriff{.NET 6.0}.
            \item Als Webframework wird \Fachbegriff{ASP.Net Core MVC} genutzt.
            \item Die Webanwendung läuft im Intranet der \acs{Icam}.
            \item Die Versionskontrolle erfolgt über \Fachbegriff{Git}.
        \end{enumerate}
    \item Benutzeroberfläche
        \begin{enumerate}
            \item Die Benutzeroberfläche wird mit Komponenten von \Fachbegriff{TelerikUI} umgesetzt.
            \item Zur Erstellung werden die Programmiersprachen \Fachbegriff{C\#} und \acs{JS} verwendet.
            \item Als Auszeichnungssprachen werden \acs{HTML} und \acs{CSS} benutzt.
            \item Die Benutzeroberfläche wird für die Betrachtung auf einem Desktop-PC entworfen.
            \item Daten werden in Tabellen strukturiert.
            \item Teilbereiche sollen über Reiter erreichbar sein.
            \item Die Suchfunktion soll einschränkbar sein.
        \end{enumerate}
    \item Geschäftslogik
        \begin{enumerate}
            \item Die Einbindung von \Fachbegriff{Services} erfolgt übe \acs{DI}.
            \item Aus der \acs{API}-Antwort heraus wird eine \acs{XML}-Datei zum Download bereitgestellt.
            \item Aus der \acs{API}-Antwort heraus wird eine \acs{PDF}-Datei zum Download bereitgestellt.
            \item Suchergebnisse werden in einem \Fachbegriff{Cache} zwischengespeichert.
        \end{enumerate}
\end{enumerate}