\subsection{Benutzerdokumentation (Ausschnitt)}
\label{app:BenutzerDoku}
% \begin{table}[htb]
% \begin{tabularx}{\textwidth}{cXX}
% \rowcolor{heading}\textbf{Symbol} & \textbf{Bedeutung global} & \textbf{Bedeutung einzeln} \\
% \includegraphicstotab[]{weather-clear.png} & Alle Module weisen den gleichen Stand auf. & Das Modul ist auf dem gleichen Stand wie das Modul auf der vorherigen Umgebung. \\
% \rowcolor{odd}\includegraphicstotab[]{weather-clear-night.png} & Es existieren keine Module (fachlich nicht möglich). & Weder auf der aktuellen noch auf der vorherigen Umgebung sind Module angelegt. Es kann also auch nichts übertragen werden. \\
% \includegraphicstotab[]{weather-few-clouds-night.png} & Ein Modul muss durch das Übertragen von der vorherigen Umgebung erstellt werden. & Das Modul der vorherigen Umgebung kann übertragen werden, auf dieser Umgebung ist noch kein Modul vorhanden. \\
% \rowcolor{odd}\includegraphicstotab[]{weather-few-clouds.png} & Auf einer vorherigen Umgebung gibt es ein Modul, welches übertragen werden kann, um das nächste zu aktualisieren. & Das Modul der vorherigen Umgebung kann übertragen werden um dieses zu aktualisieren. \\
% \includegraphicstotab[]{weather-storm.png} & Ein Modul auf einer Umgebung wurde entgegen des Entwicklungsprozesses gespeichert. & Das aktuelle Modul ist neuer als das Modul auf der vorherigen Umgebung oder die vorherige Umgebung wurde übersprungen. \\
% \end{tabularx}
% \end{table}
\begin{figure}[htb]
    \centering
    \includegraphicsKeepAspectRatio{BenutzerDoku.png}{1}
    \caption{Ansicht des ReklaTool}
\end{figure}

Benutzung des ReklaTool
\begin{enumerate}
    \item Aktenzeichen eingeben.
    \item Typ des Aktenzeichens auswählen.
    \item Option zur Auswahl der Schnellsuche. Wenn ausgewählt, werden Claimsguard-Regeln und PDF nicht mitgeschickt.
    \item Button zum Starten der Suchanfrage.
    \item Vorgang auswählen (siehe \Anhang{Screenshots}).
    \item Auswahl einer Kategorie durch klicken auf einen Reiter.
    \item In der Tabellenansicht können sortiert und gruppiert werden. Zum Gruppieren den Titel einer Spalte in die 
    Zeile darüber ziehen. Zum Sortieren auf den Spaltentitel mit dem Sortierkriterium klicken.
\end{enumerate}