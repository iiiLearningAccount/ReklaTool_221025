\subsection{Entwicklerdokumentation (Auszug)}
\label{app:EntwicklerDoku}
\subsection*{Einbindung von Services}
In der \Datei{Program.cs} werden die Services beim \Fachbegriff{Servicecontainer} des Frameworks registriert (siehe Codezeilen 7-18).
Dazu wird der Servicebuilder mit \Klasse{builder.Services} aufgerufen. Dessen Methode \Methode{.AddScoped<T>}
fügt dem Container den Service direkt, oder über ein Interface hinzu.
\lstinputlisting[language=cs, firstline=5 ,lastline=21 ,consecutivenumbers=false,
title={Program.cs -- Services}]{Listings/Program.cs}

Einbindung der Middleware.
\lstinputlisting[language=cs, firstline=31 ,lastline=36 ,consecutivenumbers=false, 
title={Program.cs -- Middleware}]{Listings/Program.cs}
\clearpage

\subsection*{Aufbau des HttpMsgService}
Die Datei \Datei{HttpMsgService.cs} beinhaltet das Interface \Klasse{IMsgService}. 
\lstinputlisting[language=cs, firstline=7 ,lastline=11 ,consecutivenumbers=false, 
title={HttpMsgService.cs -- Interface}]{Listings/HttpMsgService.cs}

Die Injektion der Services über den Konstruktor. Diese werden automatisch vom Injektor des Framework 
bereitgestellt. 
\lstinputlisting[language=cs, firstline=18 ,lastline=24 ,consecutivenumbers=false, 
title={HttpMsgService.cs -- Konstruktorinjektion}]{Listings/HttpMsgService.cs}

Services werden durch private Felder gehalten.
\lstinputlisting[language=cs, firstline=14 ,lastline=17 ,consecutivenumbers=false, 
title={HttpMsgService.cs -- Services}]{Listings/HttpMsgService.cs}

Die Anfragen an die Datenbank-API werden über die Methode \Methode{GetVorgaengeAsync} gestellt.
Das \Klasse{RequestModel} für die Anfrage wird über den \Klasse{\_requestBuilder} erstellt.
Dieser kann über ein \Fachbegriff{Fluent-Interface} konfiguriert werden.
\lstinputlisting[language=cs, firstline=25 ,lastline=30 ,consecutivenumbers=false, 
title={GetVorgaengeAsync -- Anfrage-Model}]{Listings/HttpMsgService.cs}
\clearpage

Das Anfrage-Object wird zunächst an den \Klasse{\_cache} geleitet.
Findet dieser ein entsprechendes Element, so gibt die Methode dieses zurück.
\lstinputlisting[language=cs, firstline=31 ,lastline=34 ,consecutivenumbers=false, 
title={GetVorgaengeAsync -- Cache-Abfrage}]{Listings/HttpMsgService.cs}

Falls kein Element im Cache vorhanden ist, wird zunächst eine Instanz eines \acs{HTTP}-Clients
erstellt. 
\lstinputlisting[language=cs, firstline=35 ,lastline=38 ,consecutivenumbers=false, 
title={GetVorgaengeAsync -- HTTP-Client}]{Listings/HttpMsgService.cs}

Danach wird das Anfrage-Objekt serialisiert und zusammen mit der \acs{API}-Adresse
und dem Typ der Anfrage in ein \Klasse{HttpRequestMessage}-Objekt verpackt. 
Dieses wird dem Client übergeben und an die API versendet.
\lstinputlisting[language=cs, firstline=39 ,lastline=46 ,consecutivenumbers=false, 
title={GetVorgaengeAsync -- \acs{API}-Anfrage}]{Listings/HttpMsgService.cs}

Der Inhalt der empfangenen Antwort wird deserialisiert. Danach wird
das Antwort-Ojekt zusammen mit Anfrage-Objekt direkt in den Cache geschrieben.
Abschließend gibt die Methode \Methode{GetVorgaengeAsync} ein Objekt mit den
angefragten Vorgängen zurück.
\lstinputlisting[language=cs, firstline=47 ,lastline=57 ,consecutivenumbers=false, 
title={GetVorgaengeAsync -- Antwort der API}]{Listings/HttpMsgService.cs}