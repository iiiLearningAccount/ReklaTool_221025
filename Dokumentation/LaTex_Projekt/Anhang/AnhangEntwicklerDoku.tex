\subsection{Entwicklerdokumentation (Auszug)}
\label{app:EntwicklerDoku}
\subsection*{Einbindung von Services}
In der \Datei{Program.cs} werden die Services beim \Fachbegriff{Servicecontainer} des Frameworks registriert (siehe Codezeilen 7-18).
Dazu wird der Servicebuilder mit \Klasse{builder.Services} aufgerufen. Dessen Methode \Methode{.AddScoped<T>}
fügt dem Container den Service direkt, oder über ein Interface hinzu.

\includegraphicsKeepAspectRatio{ProgramCs.png}{1}
\clearpage

\subsection*{Aufbau des HttpMsgService}
Die Datei \Datei{HttpMsgService.cs} beinhaltet das Interface \Klasse{IMsgService}. 
Die Injektion der Services über den Konstruktor wird ab Zeile 18 umgesetzt. Diese werden automatisch vom Injektor des Framework 
bereitgestellt. Services werden durch private Felder gehalten (Zeilen 14-17).

\includegraphicsKeepAspectRatio{MsgServiceKonstruktor.png}{1}\\

Die Anfragen an die Datenbank-API werden über die Methode \Methode{GetVorgaengeAsync} gestellt.
Das \Klasse{RequestModel} für die Anfrage wird über den \Klasse{\_requestBuilder} erstellt.
Dieser kann über ein \Fachbegriff{Fluent-Interface} konfiguriert werden.

\includegraphicsKeepAspectRatio{MsgServiceRequestBuilder.png}{1}\\
\clearpage
Das Anfrage-Object wird zunächst an den \Klasse{\_cache} geleitet.
Findet dieser ein entsprechendes Element, so gibt die Methode dieses zurück.

\includegraphicsKeepAspectRatio{MsgServiceCacheResponse.png}{1}\\


Falls kein Element im Cache vorhanden ist, wird zunächst eine Instanz eines \acs{HTTP}-Clients
erstellt(Zeile 38). Danach wird das Anfrage-Objekt serialisiert und zusammen mit der API-Adresse
und dem Typ der Anfrage in ein \Klasse{HttpRequestMessage}-Objekt verpackt (Zeilen 41-44). 
Dieses wird dem Client übergeben und an die API versendet.

\includegraphicsKeepAspectRatio{MsgServiceSendRequest.png}{1}\\

Der Inhalt der empfangenen Antwort wird deserialisiert(Zeile 55). Danach wird
das Antwort-Ojekt zusammen mit Anfrage-Objekt direkt in den Cache geschrieben (Zeile 58).
Abschließend gibt die Methode \Methode{GetVorgaengeAsync} ein Objekt mit den
angefragten Vorgängen zurück.

\includegraphicsKeepAspectRatio{MsgServiceReturn.png}{1}