% !TEX root = Projektdokumentation.tex

% Es werden nur die Abkürzungen aufgelistet, die mit \ac definiert und auch benutzt wurden. 
%
% \acro{VERSIS}{Versicherungsinformationssystem\acroextra{ (Bestandsführungssystem)}}
% Ergibt in der Liste: VERSIS Versicherungsinformationssystem (Bestandsführungssystem)
% Im Text aber: \ac{VERSIS} -> Versicherungsinformationssystem (VERSIS)

% Hinweis: allgemein bekannte Abkürzungen wie z.B. bzw. u.a. müssen nicht ins Abkürzungsverzeichnis aufgenommen werden
% Hinweis: allgemein bekannte IT-Begriffe wie Datenbank oder Programmiersprache müssen nicht erläutert werden,
%          aber ggfs. Fachbegriffe aus der Domäne des Prüflings (z.B. Versicherung)

% Die Option (in den eckigen Klammern) enthält das längste Label oder
% einen Platzhalter der die Breite der linken Spalte bestimmt.
\begin{acronym}[WWWWW]
	\acro{API}{Application Programming Interface}
	\acro{BPMS}{Business Process Management Software}
	\acro{bzw}[bzw.]{beziehungsweise}
	\acro{ca}[ca.]{circa}
	\acro{CG}{ClaimsGuard}
	\acro{CSS}{Cascading Style Sheets}
	\acro{DI}{Dependency Injection}
	\acro{DBMS}{Datenbankmanagementsystem}
	\acro{DTO}{Data Transfer Object}
	\acro{EPK}{Ereignisgesteuerte Prozesskette}
	\acro{ggf}[ggf.]{gegebenenfalls}
	\acro{HTML}{Hypertext Markup Language}\acused{HTML}
	\acro{HTTP}{Hypertext Transfer Protocol}\acused{HTTP}
	\acro{Icam}{IcamSystems GmbH}
	\acro{IDE}{Integrated Development Environment}
	\acro{JS}{JavaScript}
	\acro{MVC}[MVC]{Model View Controller}
	\acro{PDF}{Portable Document Format}
	\acro{SDK}{Software Development Kit}
	\acro{SME}{Subject Matter Expert}
	\acro{SPA}{Single-Page Application}
	\acro{SQL}{Structured Query Language}
	\acro{ua}[u.a.]{unter anderem}
	\acro{UML}{Unified Modeling Language}
	\acro{VID}{Vehicle Information Database}
	\acro{XML}{Extensible Markup Language}
	\acro{zb}[z.B.]{zum Beispiel}
\end{acronym}
