% !TEX root = ../Projektdokumentation.tex
\addtocontents{toc}{\protect\newpage}
\section{Implementierungsphase} 
\label{sec:Implementierungsphase}
Bei der Implementierung wurden die in der \nameref{sec:Entwurfsphase} beschriebenen Architekturvorgaben
und Programmteile umgesetzt. Begonnen wurde mit dem Erstellen des Projekts nach einer vom \Fachbegriff{Telerik-Framework}
bereitgestellten Vorlage für eine \Fachbegriff{ASP.Net Core MVC}-Anwendung. Diese liefert ein Gerüst aus
\Fachbegriff{Boilerplate-Code}, der \zB die Konfiguration vereinfachen soll.

\subsection{Konfiguration}
\label{sec:Konfiguration}
Das \Fachbegriff{ASP.Net Core MVC}-Framework bietet verschiedene Möglichkeiten zur Konfiguration.
So konnten über die Datei \Datei{appsettings.json} Information wie \acs{API}-Adresse und Daten für die
Authentifizierung in die Anwendung gegeben werden. Diese wurden durch das Interface \Klasse{IConfiguration}
via \acs{DI} den anderen Services zur Verfügung gestellt. \\
In der Datei \Datei{Program.cs} werden alle Services mit entsprechender Lebensdauer (scoped, transient oder singleton)
registriert.\footnote{Service-Lifetimes in der Microsoft-Dokumentation:\\
\url{https://learn.microsoft.com/en-us/dotnet/core/extensions/dependency-injection\#service-lifetimes},\\
letzter Zugriff: 28.10.2022}
Für das Projekt wurden diese als \Fachbegriff{scoped} angemeldet. Das bedeutet, dass der Service immer für die Dauer einer
\acs{API}-Anfrage zur Verfügung steht. Eine Beschreibung, wie Services in das Programm eingebunden sind, findet sich im 
\Anhang{app:EntwicklerDoku}.\\
Nach der Registrierung der Dienste wurden in der \Datei{Program.cs} verschiedene \Fachbegriff{Middleware}-An\-wen\-dung\-en 
registriert und konfiguriert. Diese können vor \acs{bzw} nach dem Programmcode die Anfrage oder die Antwort auswerten 
und bearbeiten. Ein Typisches Beispiel ist die Autorisierung eines Benutzers.

\subsection{Authentifizierung}
\label{sec:Authentifizierung}
Für die Authentifizierung und Autorisierung setzt die \acs{Icam} auf den Standard \Fachbegriff{OAuth2}.
Zur Umsetzung wurde dabei auf die Software \Fachbegriff{Keycloak} zurückgegriffen. Dazu hat die IT-Abteilung einen Client 
angelegt. Die Zugangsdaten mussten dann der Anwendung mitgeteilt werden. Durch diverse Softwarebibliotheken 
wurde der Authentifizierungsprozess in die \Fachbegriff{Middleware} von \Fachbegriff{ASP.NET Core} eingefügt. Durch das Attribut 
\Methode{[Authorize]} wurde der \Klasse{VorgangController} vor unberechtigtem Zugriff geschützt. Die Verwendung 
des \Fachbegriff{Keycloak} und \Fachbegriff{OAuth2} sorgte für eine schnellere Entwicklung, da die Benutzerveraltung nicht mehr implementiert
werden musste. Diese wird extern von der IT-Abteilung über das an den \Fachbegriff{Keycloak} angeschlossene 
\Fachbegriff{Active Directory} verwaltet. Durch diesen Vorgang wurde die Implementierung der Nutzerauthentifizierung 
verkürzt (siehe dazu \ref{sec:SollIstVergleich} \nameref{sec:SollIstVergleich}).

\subsection{Implementierung der Datenstrukturen}
\label{sec:ImplementierungDatenstrukturen}
% \begin{itemize}
% 	\item Beschreibung der angelegten Datenbank (\zB Generierung von \acs{SQL} aus Modellierungswerkzeug oder händisches Anlegen), \acs{XML}-Schemas \usw.
% \end{itemize}
Zur Erstellung der einzelnen \ac{DTO}s wurden die \acs{XML}-Dateien deserialisiert, 
über die der Informationsaustausch mit der Datenbank-\acs{API} realisiert wird. 
Dabei handelte es sich um jeweils ein Anfrage- und ein Antwortformat. Diese wurden 
von der Fachabteilung zur Verfügung gestellt.\\
Das Anfrage-\Fachbegriff{Model} beinhaltet das jeweils gewünschte Aktenzeichen und 
eine Information über die Anwendung eines Suchfilters, welcher die zurückgelieferten 
Daten einschränkt.\\
Das Antwort-\Fachbegriff{Model} wurde bei der Deserialisierung bereits automatisch 
mit den notwendigen Annotationen versehen, um einen reibungslosen Austausch mit 
dem \acs{XML}-Format zu garantieren (siehe \Anhang{app:DTO}). Zur Datenbindung an 
die einzelnen Komponenten der Benutzeroberfläche wurden weitere \Fachbegriff{Viewmodels} 
erstellt, welche das Mapping auf das Antwort-\Fachbegriff{Model} ermöglichen.\\
Für das \Fachbegriff{Caching} wurde ein eigenes \acs{DTO} umgesetzt, welches das 
Anfrage- und Antwort-\Fachbegriff{Model} enthält. Dieses wird beim Lesen und Schreiben 
der \Fachbegriff{Cache}-Dateien serialisiert \bzw deserialisiert. Über das \Fachbegriff{Hashing} 
des Anfrage-Objekts sind die Dateien eindeutig identifizierbar.

\subsection{Implementierung der Geschäftslogik}
\label{sec:ImplementierungGeschaeftslogik}
% \begin{itemize}
% 	\item Beschreibung des Vorgehens bei der Umsetzung/Programmierung der entworfenen Anwendung.
% 	\item \Ggfs interessante Funktionen/Algorithmen im Detail vorstellen, verwendete Entwurfsmuster zeigen.
% 	\item Quelltextbeispiele zeigen.
% 	\item Hinweis: Wie in Kapitel~\ref{sec:Einleitung}: \nameref{sec:Einleitung} zitiert, wird nicht ein lauffähiges Programm bewertet, sondern die Projektdurchführung. Dennoch würde ich immer Quelltextausschnitte zeigen, da sonst Zweifel an der tatsächlichen Leistung des Prüflings aufkommen können.
% \end{itemize}
% \paragraph{Beispiel}
% Die Klasse \texttt{Com\-par\-ed\-Na\-tu\-ral\-Mo\-dule\-In\-for\-ma\-tion} findet sich im \Anhang{app:CNMI}.  
Laut \ref{sec:Anwendungsfaelle} \nameref{sec:Anwendungsfaelle} ist der Hauptanwendungsfall
das Abfragen der Datenbank-\acs{API} mithilfe des \Fachbegriff{ReklaTool}. Aus diesem wurde in 
der Entwurfsphase ein Sequenz-Diagramm erstellt (\Vgl \Anhang{app:Sequenzdiagramm}).\\
Nach der Initialisierung der Projektdateien, wurden die aus den Diagrammen ersichtlichen Serviceklassen
und die dazugehörigen Interfaces implementiert. \\Danach konnten diese in der Datei \Datei{Program.cs}
über den Aufruf \Methode{AddScoped} der \Klasse{builder.Services}-Instanz dem \acs{IOC}-Container 
hinzugefügt werden und standen somit dem Injektor des Frameworks zur Verfügung. \\Danach 
wurden die Services wie im Klassendiagramm \Anhang{app:Klassendiagramm} miteinander verknüpft.
Dieses passierte über \Fachbegriff{Konstruktor-Injektion}. Die Implementierung von \Fachbegriff{.NET Core} 
sieht ein Injizieren mittels Konstruktorparameter vor. Gegenüber der \Fachbegriff{Property-Injektion} gibt es
den Vorteil, dass beim Instantiieren der Klasse bereits fehlende Abhängigkeiten auffallen.
\footnote{Dependency Injection in der Microsoft-Dokumentation:\\ 
\url{https://learn.microsoft.com/de-de/dotnet/core/extensions/dependency-injection}, \\
letzter Zugriff: 28.10.2022}
\clearpage

\subsection{Exception Handling}
\label{sec:ExceptionHandling}
Zum Abfangen möglicher Fehler, die \zB bei der Deserialisierung der Antwortdaten 
der \acs{API} entstehen können, wurde so nah wie möglich am Einstiegspunkt in das 
Programm das \Fachbegriff{Exception Handling} implementiert. Der Code der Methoden 
innerhalb des Controllers \Datei{VorgangController.cs} wurde jeweils durch einen
\Fachbegriff{try/catch}-Block umschlossen (siehe \Anhang{app:ExceptionHandling}). 
Die Fehlermeldung der Ausnahme (\Fachbegriff{Exception}) wird an die Programmkonsole ausgegeben 
und kann dort zur Fehleranalyse genutzt werden. Die Methode gibt eine Fehlernachricht an den 
\Fachbegriff{Client} zurück. Das \Fachbegriff{Frontend-Framework} informiert danach
den Anwender darüber, dass seine Anfrage nicht erfolgreich war.\\
Im Rahmen der Abgrenzung des Projekts wurde das \Fachbegriff{Logging} nur rudimentär
implementiert. Für spätere Weiterentwicklungen ist vorgesehen, die Fehlernachrichten
an die hausinterne Instanz der Software \Fachbegriff{Sentry.io} weiterzuleiten.
Dadurch bietet sich die Möglichkeit der Benachrichtigung von Administratoren und
\Fachbegriff{DevOps}.

\subsection{Implementierung der Benutzeroberfläche}
\label{sec:ImplementierungBenutzeroberflaeche}
% \begin{itemize}
% 	\item Beschreibung der Implementierung der Benutzeroberfläche, falls dies separat zur Implementierung der Geschäftslogik erfolgt (\zB bei \acs{HTML}-Oberflächen und Stylesheets).
% 	\item \Ggfs Beschreibung des Corporate Designs und dessen Umsetzung in der Anwendung.
% 	\item Screenshots der Anwendung
% \end{itemize}
% \paragraph{Beispiel}
% Screenshots der Anwendung in der Entwicklungsphase mit Dummy-Daten befinden sich im \Anhang{Screenshots}.
Bereits in \ref{sec:Benutzeroberflaeche}: \nameref{sec:Benutzeroberflaeche} wurde festgelegt, dass die Benutzeroberfläche
als Webseite in einem Browser ausgeführt werden soll.\\
Die von \Fachbegriff{TelerikUI} bereitgestellte Vorlage für eine \acs{MVC}-Anwendung verlangt
die Implementierung der \Fachbegriff{View} mittels spezieller \Fachbegriff{.cshtml}-Dateien.
Diese erlauben, durch definierte Schlüsselwörter, das Verwenden von \Fachbegriff{C\#}-Syntax in einem
\acs{HTML}-Dokument. Beides zusammen wird innerhalb des \Fachbegriff{ASP.NET}-Ökosystems als \Fachbegriff{Razor}-Syntax
bezeichnet.\footnote{Razor-Syntax in der Microsoft-Dokumentation:\\
\url{https://learn.microsoft.com/de-de/aspnet/core/mvc/views/razor?view=aspnetcore-6.0}, \\
letzter Zugriff: 28.10.2022}\\ 
Die Oberflächenkomponenten, \zB Buttons, Suchfelder oder Tabellen, werden daher auch in \Fachbegriff{C\#} 
definiert und konfiguriert. Siehe dazu \Anhang{app:ListingView}.\\
Wie bei den meisten Webseiten wurde auf die Verwendung der Programmiersprache \acs{JS} im Zusammenspiel mit \acs{AJAX} 
zurückgegriffen, um die ereignisbasierte Interaktion mit dem Nutzer zu gestalten.
\footnote{AJAX in der Mozilla-Dokumentation:\\
\url{https://developer.mozilla.org/en-US/docs/Web/Guide/AJAX},\\
letzter Zugriff: 28.10.2022}
\clearpage