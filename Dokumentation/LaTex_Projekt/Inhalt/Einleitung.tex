% !TEX root = ../Projektdokumentation.tex
\section{Einleitung}
\label{sec:Einleitung}


\subsection{Projektumfeld} 
\label{sec:Projektumfeld}
%\begin{itemize}
%	\item Kurze Vorstellung des Ausbildungsbetriebs (Geschäftsfeld, Mitarbeiterzahl \usw)\\
%	\item Wer ist Auftraggeber/Kunde des Projekts?
%\end{itemize}
Die IcamSystems GmbH bietet Softwarelösungen im Bereich der Kfz-Schadensregulierung an.
Sie ist Teil eines Firmenverbundes mit ca. 160 Mitarbeitern. Zu den Kunden zählen unter
anderem Versicherungsunternehmen, Sachverständigenorganisationen, Prüfdienstleister und
Schadensteuerungsgesellschaften.\\
Zum Leistungsspektrum gehört die automatisierte Prüfung von Gutachten und
Kostenvoranschlägen, welche durch Gutachter oder Partnerwerkstätten erstellt wurden.\\
Das Projekt wurde intern durch das Team Reklamationsbearbeitung aus der Abteilung Prozessautomatisierung
in Auftrag gegeben. Die Fachabteilung setzt unternehmensspezifische Prozesse \ua mittels
\ac{BPMS} um.	


\subsection{Projektziel} 
\label{sec:Projektziel}
%\begin{itemize}
%	\item Worum geht es eigentlich?
%	\item Was soll erreicht werden?
%\end{itemize}	
Ziel ist eine Webanwendung mit einer funktionalen Benutzeroberfläche, welche Daten von
einem Backend-Service abfragt, um diese strukturiert und übersichtlich darzustellen.
Reklamationen sollen so schneller bearbeitet werden können und die händische Suche nach
allen Teilinformationen überflüssig machen.

% Projektanforderungen
% \begin{itemize}
% \item Eingabe von vorgangsspezifischen Aktenzeichen als Suchwörter
% \item Möglichkeit einer Schnellsuche ohne Einzelprüfbericht und Regelauslösungen
% \item Abfrage von externen Datenbanken und Aufbereitung der Daten zum Vorgang
% \item Auswahl eines Vorgangs zum gesuchten Aktenzeichen
% \item Aufbereitung und Anzeige des ausgewählten Vorgangs in folgende Bereiche:
% 	\begin{itemize}
% 		\item Allgemeine Übersicht zur Kalkulation
% 		\item Übersicht der kalkulierten Positionen
% 		\item Informationen aus der Fahrzeugdatenbank VID
% 		\item Ausgelöste Regeln für den Vorgang
% 	\end{itemize}
% \end{itemize}
% \begin{itemize}
% \item Download des Einzelprüfberichts zum Vorgang
% \item Download der Kalkulation als strukturierte Datei
% \item Authentifizierung per Identity Server mittels OAuth2
% \item Suche darf nicht länger als 7 Sekunden ohne Reaktion bleiben
% 	\begin{itemize}
% 		\item Abfrage aller Daten dauert in der Regel über 30 Sekunden
% 		\item Lösungsansatz: Bereitstellen von Teilergebnissen
% 	\end{itemize}
% \end{itemize}

	



\subsection{Projektbegründung} 
\label{sec:Projektbegruendung}
%\begin{itemize}
%	\item Warum ist das Projekt sinnvoll (\zB Kosten- oder Zeitersparnis, weniger Fehler)?
%	\item Was ist die Motivation hinter dem Projekt?
%\end{itemize}	
Die Vorgangsprüfung erfolgt über das automatisierte Prüfregelwerk \ac{CG}.
Darin prüft ein individuell erstelltes Regelwerk die, z.B. von Werkstätten, erstellten
Kostenvoranschläge und Gutachten. Für die individuellen Parameter jedes Fahrzeugtyps greift
der \acs{CG} auf die recherchierten Fahrzeugdaten der \ac{VID} zurück. 
Die \acs{VID} bietet für jedes Kfz-Bauteil Informationen zur Beschaffenheit und Verarbeitung.\\
Bei der Plausibilitätsprüfung durch den \acs{CG} entsteht ein detaillierter Prüfbericht, in dem die
angewendeten Prüfregeln gelistet sind und ggf. Diskrepanzen aufgeführt werden.\\
Sollte der Kunde das Prüfergebnis beanstanden, so hat er die Möglichkeit eine Reklamation an
die IcamSystems GmbH zu senden.\\
Aktuell können nur die Projektverantwortlichen selbst, mittels des jeweiligen Aktenzeichens der
Reklamation, eine händische Suche in mehreren Datenbanken und dem Regelwerk des \acs{CG}
durchführen, um den Vorgang nachzuvollziehen. Es soll für die Bewertung des Vorgangs
ersichtlich sein, welche Regeln ausgelöst wurden.\\
Dieser Prozess ist aufgrund des hohen Anteils manueller Arbeit zeitaufwändig und fehleranfällig.
Weiterhin kann dieser, wegen der Komplexität der Datenbanksysteme, nur von Experten durchgeführt werden.
Durch das ReklaTool sollen die Suchanfragen vereinfacht und standardisiert werden, was wiederum 
eine Erweiterung des Nutzerkreises möglich macht.

\subsection{Projektschnittstellen} 
\label{sec:Projektschnittstellen}
%\begin{itemize}
%	\item Mit welchen anderen Systemen interagiert die Anwendung (technische Schnittstellen)?
%	\item Wer genehmigt das Projekt \bzw stellt Mittel zur Verfügung? 
%	\item Wer sind die Benutzer der Anwendung?
%	\item Wem muss das Ergebnis präsentiert werden?
%\end{itemize}
\textbf{Technische Schnittstellen}\\
In der Abteilung Prozessautomatisierung wurde mittels \acs{BPMS} ein Webservice erstellt.
Dieser trägt Daten aus Produktiv-, Archiv- und Testdatenbanken zusammen.
Die Daten werden mit den fahrzeugspezifischen Daten aus der \acs{VID} ergänzt.
Hinzu kommen die im Vorgang ausgelösten Regeln des \acs{CG}.
Dieses Datenpaket wird der Webanwendung in strukturierter Form zur Verfügung gestellt.

\textbf{Verantwortlichkeit}\\
Das Projekt wird durch die Abteilung Projektmanagement begleitet.

\textbf{Benutzer der Anwendung}\\
Anwender sind zum jetzigen Stand die Projektverantwortlichen des \acs{CG}.
Aufgrund der bisherigen Komplexität der Datenbankabfragen waren bisher nur \ac{SME}
für die Abfragen zuständig. Durch die Vereinfachung dieses Vorgangs, ist eine Erweiterung des Nutzerkreises denkbar.

\textbf{Endabnahme}\\
Das Ergebnis des Projekts wird in der IT-Abteilung und von den \acs{SME} getestet und abgenommen.
Die Dokumentation wird der Projektmanagementabteilung übergeben.

\subsection{Projektabgrenzung} 
\label{sec:Projektabgrenzung}
%\begin{itemize}
%	\item Was ist explizit nicht Teil des Projekts (\insb bei Teilprojekten)?
%\end{itemize}
Der Webservice für den Datenbankzugriff wurde via \acs{BPMS} in der Abteilung Prozessautomatisierung realisiert und bereitgestellt. 
Er existiert unabhängig vom ReklaTool.
