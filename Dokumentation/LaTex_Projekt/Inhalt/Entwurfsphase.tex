% !TEX root = ../Projektdokumentation.tex
\section{Entwurfsphase} 
\label{sec:Entwurfsphase}
Dieser Abschnitt behandelt alle wichtigen Entscheidungen bezüglich der Grundprinzipien, Techniken und Technologien zum
Erstellen der Webanwendung.

\subsection{Zielplattform}
\label{sec:Zielplattform}
% \begin{itemize}
% 	\item Beschreibung der Kriterien zur Auswahl der Zielplattform (\ua Programmiersprache, Datenbank, Client/Server, Hardware).
% \end{itemize}
Wie unter \ref{sec:Projektziel} \nameref{sec:Projektziel} erwähnt, soll eine
Webanwendung erstellt werden. Aus der IT-Abteilung kamen dazu Informationen zu bereits 
in anderen Projekten eingesetzten Technologien.\\
Neue Software wird dort überwiegend in der Programmiersprache C\# umgesetzt. 
Für diese existiert bereits breites Wissen bei den Mitarbeitern.
Daher konnten vorhandene Ressourcen und Lizenzen genutzt werden.
Um das Endprodukt besser im Team wartbar und erweiterbar zu halten, hat sich der Autor 
ebenfalls für C\# in der .NET-Umgebung von Microsoft entschieden.\\
Unter weiterer Rücksprache mit der Softwareabteilung wurden verschiedene Ansätze zur
Umsetzung einer Webanwendung durchgesprochen. Besonders wurde hier zwischen client- und serverbasierten
Anwendungen unterschieden. Da die Benutzerinteraktion eher gering ausfallen würde, wurde sich
für die nicht ganz so dynamische Servervariante entschieden.\\
Die Wahl viel hier auf das Webframework \Fachbegriff{ASP.Net Core MVC}, zu dem es schon gute Erfahrungen im Team gab.
Es bietet bereits Projektvorlagen mit der grundlegenden Programmstruktur und Möglichkeiten zur
Konfiguration an. 

\subsection{Architekturdesign}
\label{sec:Architekturdesign}
% \begin{itemize}
% 	\item Beschreibung und Begründung der gewählten Anwendungsarchitektur (\zB \acs{MVC}).
% 	\item \Ggfs Bewertung und Auswahl von verwendeten Frameworks sowie \ggfs eine kurze Einführung in die Funktionsweise des verwendeten Frameworks.
% \end{itemize}
Wie bereits an der Wahl des Webframworks erkennbar ist, basiert die Anwendung auf dem \acs{MVC}-Entwurfsmuster.
Dieses beinhaltet die Teile \textbf{M}odel, \textbf{V}iew und \textbf{C}ontroller -- also Datenabbildung, 
Benutzeroberfläche und Vermittlung. Durch die klare Trennung in diese drei Bereiche vermindern sich
die Abhängigkeiten untereinander. Implementierungen dieser Teile können leichter verändert oder sogar ausgetauscht werden.
Ein weiterer Vorteil ist der gut nachvollziehbare Datenfluss innerhalb der Anwendung.\\
Dieser Kern des Programms wird durch Services erweitert, welche jeweils eine spezielle Aufgabe haben.
Dazu bietet das Framework einen Injektor, mit dem das Prinzip der \Fachbegriff{Inversion of Controll} via 
\Fachbegriff{Dependency Injection} umgesetzt wird. Die daraus folgende lose Verbindung der einzelnen Module
trägt weiter zur besseren Wartbarkeit bei. Dabei müssen sich die Teile des Programms nicht um die Erstellung von
Instanzen ihrer Abhängigkeiten kümmern, sondern Bekommen diese vom Injektor übergeben. Dieser übernimmt
das komplette Management der erstellten Objekte und kann diese, je nach Lebensdauer, \ggfs zerstören, sollten sie nicht mehr 
gebraucht werden.\\
Zur Kommunikation mit dem Datenbankserver ist ein Messagingservice nötig. Um die Performance von
mehrfach aufgerufenen Vorgängen zu erhöhen, wird dieser um ein \Fachbegriff{Caching}-Modul erweitert.
% Die \Fachbegriff{View} wird mittels des Framework \Fachbegriff{Telerik UI} von \Fachbegriff{Progress} umgesetzt, welches eine gute Integration
% in das \Fachbegriff{ASP.Net-Core}-Framework bietet.\\
Das Model wird mittels verhaltensloser Datenklassen erstellt. 
% Diese ergeben sich aus der (De-)Serialisierung von XML-Dateien, über die die
% Kommunikation mit dem Datenbank-Webservice erfolgt. 

\subsection{Entwurf der Benutzeroberfläche}
\label{sec:Benutzeroberflaeche} 
% \begin{itemize}
% 	\item Entscheidung für die gewählte Benutzeroberfläche (\zB GUI, Webinterface).
% 	\item Beschreibung des visuellen Entwurfs der konkreten Oberfläche (\zB Mockups, Menüführung).
% 	\item \Ggfs Erläuterung von angewendeten Richtlinien zur Usability und Verweis auf Corporate Design.
% \end{itemize}
Um den einfachen Zugriff von beliebigen Rechnern im Firmenintranet zu gewährleisten,
fiel die Entscheidung auf eine Umsetzung mit Weboberfläche. Dies sollte in den gängigsten Browser geöffnet werden können.
Da die Anwendung lediglich auf Desktopcomputern zum Einsatz kommen sollte, wurde bewusst auf Responsivität verzichtet
und der Fokus auf eine funktionale Gestaltung gelegt.\\
Über mehrere Korrekturschleifen wurde zusammen mit der Fachabteilung der Entwurf für eine Benutzeroberfläche
fertiggestellt. Mithilfe von \Fachbegriff{Mockups} wurden die Bedürfnisse der Endbenutzer so gut wie möglich nachempfunden.
Diese befinden sich im \Anhang{app:Entwuerfe}.\\
Die Webseite soll in der Hauptsache aus einer Suchleiste für Nutzeranfragen und einem Bereich zur Darstellung der Vorgangsdaten
bestehen. Eine Checkbox soll die Möglichkeit geben, den Umfang der Suche einzuschränken.\\
Die Vorgangsdaten werden nach erfolgreicher Suche in einem Bereich angezeigt, der die einzelnen Daten mittels Registerkarten
in Kategorien unterteilt. Zur strukturierten Darstellung werden filterbare Tabellen genutzt. Weiterhin soll es möglich sein,
PDF-Dateien anzuzeigen und zum Download anzubieten.


\subsection{Datenmodell}
\label{sec:Datenmodell}
% \begin{itemize}
% 	\item Entwurf/Beschreibung der Datenstrukturen (\zB \acs{ERM} und/oder Tabellenmodell, \acs{XML}-Schemas) mit kurzer Beschreibung der wichtigsten (!) verwendeten Entitäten.
% \end{itemize}
Die Datenmodelle für sowohl Anfragen, als auch Serverantworten wurden durch Deserialisierung von XML-Dateien erstellt. 
Diese wurden vom Fachbereich bereitgestellt.\\
Im \Anhang{app:Datenmodell} befindet sich dazu eine Übersicht. 

% \paragraph{Beispiel}
% In \Abbildung{ER} wird ein \ac{ERM} dargestellt, welches lediglich Entitäten, Relationen und die dazugehörigen Kardinalitäten enthält. 

% \begin{figure}[htb]
% \centering
% \includegraphicsKeepAspectRatio{ERDiagramm.pdf}{0.6}
% \caption{Vereinfachtes ER-Modell}
% \label{fig:ER}
% \end{figure} 


\subsection{Geschäftslogik}
\label{sec:Geschaeftslogik}
% \begin{itemize}
% 	\item Modellierung und Beschreibung der wichtigsten (!) Bereiche der Geschäftslogik (\zB mit Kom\-po\-nen\-ten-, Klassen-, Sequenz-, Datenflussdiagramm, Programmablaufplan, Struktogramm, \ac{EPK}).
% 	\item Wie wird die erstellte Anwendung in den Arbeitsfluss des Unternehmens integriert?
% \end{itemize}
% \paragraph{Beispiel}
% Ein Klassendiagramm, welches die Klassen der Anwendung und deren Beziehungen untereinander darstellt kann im \Anhang{app:Klassendiagramm} eingesehen werden.
% \Abbildung{Modulimport} zeigt den grundsätzlichen Programmablauf beim Einlesen eines Moduls als \ac{EPK}.
% \begin{figure}[htb]
% \centering
% \includegraphicsKeepAspectRatio{modulimport.pdf}{0.9}
% \caption{Prozess des Einlesens eines Moduls}
% \label{fig:Modulimport}
% \end{figure}
Unter den Anwendungsfällen ist das Abfragen der Daten vom Datenbankwebservice der wichtigste.
Anhand diesem wird ein Großteil des Programmablaufs nachvollziehbar und kann im Sequenzdiagramm
im \Anhang{app:Sequenz} betrachtet werden.\\
Im gezeigten Fall ruft die Mitarbeiterin der Fachabteilung über ihren Browser das \Fachbegriff{ReklaTool} auf.
Dort gibt sie über das Suchfeld das Aktenzeichen zum gesuchten Vorgang ein. Da die Aktenzeichen kein festes 
Format besitzen muss zusätzlich deren Art (Dateiname, Vorgangsnummer, Schadennummer oder Auftragsnummer)
ausgewählt werden. Um die Suche einzuschränken kann ein Haken gesetzt werden, ob Dokumente zum Vorgang mitgeliefert werden sollen.
Nach Betätigung der Suchen-Taste wird eine \Fachbegriff{GET}-Anfrage mit Aktenzeichen und Filter an den 
\Fachbegriff{ReklaTool}-Server gestellt. Dort bietet der \Klasse{VorgangController} Endpunktmethoden für http-Anfragen an.
Über den \Klasse{UiEndpointService} werden die Anfragen an den \Klasse{HttpMsgService} durchgereicht.
Dieser erstellt mittels des \Klasse{RequestBuilder} ein Anfrage-Objekt.\\
Die erzeugte Anfrage wird zunächst durch den \Klasse{ResponseCacheService} mit den bereits zwischengespeicherten
Vorgängen verglichen. Ist eine Datei zum Aktenzeichen vorhanden, so wird diese an den \Klasse{UiEndpointService} 
zurückgegeben. Andernfalls wird nun eine Anfrage an den \Fachbegriff{Datenbank-Webservice} gestellt.
Das zurückgelieferte XML-Dokument wird deserialisiert und zusammen mit den Anfrageinformationen im \Fachbegriff{Cache} abgelegt.
Danach geht auch dieses zurück an den \Klasse{UiEndpointService}.\\
Hier werden die Daten, je nach Client-Anfrage, vom kompletten Vorgangs-Model auf einzelne \Fachbegriff{View-Models} gemappt.
Zurück im \Klasse{VorgangController} wird nun die Server-Response zurück an den Client gesendet und die Daten
an die entsprechenden Komponenten gebunden, wo sie für den Nutzer sichtbar werden.

\subsection{Maßnahmen zur Qualitätssicherung}
\label{sec:Qualitaetssicherung}
% \begin{itemize}
% 	\item Welche Maßnahmen werden ergriffen, um die Qualität des Projektergebnisses (siehe Kapitel~\ref{sec:Qualitaetsanforderungen}: \nameref{sec:Qualitaetsanforderungen}) zu sichern (\zB automatische Tests, Anwendertests)?
% 	\item \Ggfs Definition von Testfällen und deren Durchführung (durch Programme/Benutzer).
% \end{itemize}


\subsection{Pflichtenheft/Datenverarbeitungskonzept}
\label{sec:Pflichtenheft}
% \begin{itemize}
% 	\item Auszüge aus dem Pflichtenheft/Datenverarbeitungskonzept, wenn es im Rahmen des Projekts erstellt wurde.
% \end{itemize}
% \paragraph{Beispiel}
% Ein Beispiel für das auf dem Lastenheft (siehe Kapitel~\ref{sec:Lastenheft}: \nameref{sec:Lastenheft}) aufbauende Pflichtenheft ist im \Anhang{app:Pflichtenheft} zu finden.
