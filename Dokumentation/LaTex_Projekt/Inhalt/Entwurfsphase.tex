% !TEX root = ../Projektdokumentation.tex
\section{Entwurfsphase} 
\label{sec:Entwurfsphase}
Dieser Abschnitt behandelt alle wichtigen Entscheidungen bezüglich der Grundprinzipien, Techniken und Technologien zum
Erstellen der Webanwendung. Ein Überblick über die Einbettung des ReklaTool in seine
Umgebung befindet sich im Anhang \Anhang{app:Systemkontext}.

\subsection{Zielplattform}
\label{sec:Zielplattform}
% \begin{itemize}
% 	\item Beschreibung der Kriterien zur Auswahl der Zielplattform (\ua Programmiersprache, Datenbank, Client/Server, Hardware).
% \end{itemize}
Wie unter \ref{sec:Projektziel}: \nameref{sec:Projektziel} erwähnt, 
soll eine Webanwendung erstellt werden. Aus der IT-Abteilung kamen dazu Informationen zu  
Technologien, die in anderen Projekten eingesetzt wurden.\\
Neue Software wird dort überwiegend in der Programmiersprache \Fachbegriff{C\#} umgesetzt. 
Dadurch existiert für diese umfangreiches Wissen bei den Mitarbeitern.
Vorhandene Ressourcen und Lizenzen konnten für das Projekt genutzt werden.
Um das Endprodukt besser im Team wartbar und erweiterbar zu halten, hat sich der Autor 
ebenfalls für \Fachbegriff{C\#} in der \Fachbegriff{.NET}-Umgebung von Microsoft entschieden.\\
Unter weiterer Rücksprache mit der Softwareabteilung wurden verschiedene Ansätze zur
Umsetzung einer Webanwendung durchgesprochen. Besonders wurde hier zwischen client- und serverbasierten
Anwendungen unterschieden. Da die Benutzerinteraktion eher gering ausfallen würde, wurde sich
für die Servervariante entschieden. Diese ist gegenüber einer \ac{SPA} auf einem Client weniger agil und die 
Reaktionszeit ist eingeschränkt. Durch Nutzung des \Fachbegriff{TelerikUI}-Framework und der damit
verbundenen Bindung der Daten an die Komponenten der Oberfläche entsteht eine Art Mischform der zuvor genannten Varianten.\\
Die Wahl des Webframework fiel auf \Fachbegriff{ASP.Net Core MVC}, zu dem es schon gute Erfahrungen im Team gab.
Es bietet bereits Projektvorlagen mit der grundlegenden Programmstruktur und Möglichkeiten zur
Konfiguration an. 

\subsection{Architekturdesign}
\label{sec:Architekturdesign}
% \begin{itemize}
% 	\item Beschreibung und Begründung der gewählten Anwendungsarchitektur (\zB \acs{MVC}).
% 	\item \Ggfs Bewertung und Auswahl von verwendeten Frameworks sowie \ggfs eine kurze Einführung in die Funktionsweise des verwendeten Frameworks.
% \end{itemize}
Wie bereits an der Wahl des Webframworks erkennbar ist, basiert die Anwendung auf dem \acs{MVC}-Entwurfsmuster.
Dieses beinhaltet die Teile \textbf{M}odel, \textbf{V}iew und \textbf{C}ontroller -- also Datenabbildung, 
Benutzeroberfläche und Vermittlung (siehe \Anhang{app:MVC}). Durch die klare Trennung in diese drei Bereiche vermindern sich
die Abhängigkeiten untereinander. Implementierungen dieser Teile können leichter verändert oder sogar ausgetauscht werden.
Ein weiterer Vorteil ist der gut nachvollziehbare Datenfluss innerhalb der Anwendung.\\
Dieser Kern des Programms wird durch Services erweitert, welche jeweils eine spezielle Aufgabe haben.
Dazu bietet das Framework einen Injektor, mit dem das Prinzip der \ac{IOC} via 
\ac{DI} umgesetzt wird. Dabei müssen sich die Teile des Programms nicht um die Erstellung von
Instanzen ihrer Abhängigkeiten kümmern, sondern Bekommen diese vom Injektor übergeben. Dieser übernimmt
das komplette Management der erstellten Objekte und kann diese, je nach Lebensdauer, \acs{ggf} entfernen, sollten sie nicht mehr 
gebraucht werden.\\ 
Die daraus folgende lose Verbindung der einzelnen Module trägt 
weiter zur besseren Wartbarkeit der Anwendung bei.\\
Zur Steuerung der Kommunikation mit der Datenbank-\acs{API} wird ein eigener Service entwickelt. Dieser übermittelt
Anfragen und stellt die Antworten der Anwendung zur Verfügung. Um die Performance von
mehrfach aufgerufenen Vorgängen zu erhöhen, wird der Dienst um einen \Fachbegriff{Caching}-Service erweitert.
Beim \Fachbegriff{Caching} werden die erhaltenen Daten in Dateien gespeichert, welche bestimmten Abfragen
zugeordnet und später wieder eingelesen werden können.
Das \Fachbegriff{Model} wird mittels verhaltensloser Datenklassen (\acs{DTO}s) erstellt. 

\subsection{Entwurf der Benutzeroberfläche}
\label{sec:Benutzeroberflaeche} 
% \begin{itemize}
% 	\item Entscheidung für die gewählte Benutzeroberfläche (\zB GUI, Webinterface).
% 	\item Beschreibung des visuellen Entwurfs der konkreten Oberfläche (\zB Mockups, Menüführung).
% 	\item \Ggfs Erläuterung von angewendeten Richtlinien zur Usability und Verweis auf Corporate Design.
% \end{itemize}
Um den einfachen Zugriff von beliebigen Rechnern im Firmenintranet zu gewährleisten,
fiel die Entscheidung auf eine Umsetzung mit Weboberfläche. Diese soll in dem vom Unternehmen genutzten Browser angezeigt werden können.
Da die Anwendung lediglich auf Desktopcomputern zum Einsatz kommen wird, wurde bewusst auf Anpassungen für mobile Geräte verzichtet.
Der Fokus wird dabei auf eine funktionale Gestaltung gelegt.\\
Über mehrere Korrekturschleifen wurde zusammen mit der Fachabteilung der Entwurf für eine Benutzeroberfläche
fertiggestellt. Mithilfe von \Fachbegriff{Mockups} wurden die Bedürfnisse der Endbenutzer so gut wie möglich nachempfunden.
Diese befinden sich im \Anhang{app:Entwuerfe}.\\
Die Webseite soll in der Hauptsache aus einer Suchleiste für Nutzeranfragen und einem Bereich zur Darstellung der Vorgangsdaten
bestehen. Eine Checkbox soll die Möglichkeit geben, den Umfang der Suche einzuschränken.\\
Die Vorgangsdaten werden nach erfolgreicher Suche in einem Bereich angezeigt, der die einzelnen Daten mittels Registerkarten
in Kategorien unterteilt. Zur strukturierten Darstellung werden filterbare Tabellen genutzt. Diese werden durch das
Framework \Fachbegriff{TelerikUI} eines Drittanbieters bereitgestellt und bieten vielfältige Einstellungen zum Filtern,
Sortieren und Gruppieren an. Eine Lizenz ist hierfür bereits im Unternehmen vorhanden.
Weiterhin soll es möglich sein, eine \acs{PDF}- oder \acs{XML}-Datei mit den Vorgangsdaten anzuzeigen \acs{bzw} diese zum 
Download anzubieten (siehe \ref{sec:Anwendungsfaelle}: \nameref{sec:Anwendungsfaelle}).

\subsection{Datenmodell}
\label{sec:Datenmodell}
% \begin{itemize}
% 	\item Entwurf/Beschreibung der Datenstrukturen (\zB \acs{ERM} und/oder Tabellenmodell, \acs{XML}-Schemas) mit kurzer Beschreibung der wichtigsten (!) verwendeten Entitäten.
% \end{itemize}
Die Datenmodelle für Anfragen an die Datenbank-\acs{API} und deren Antwort basieren auf den versendeten \acs{XML}-Dateien. 
Diese wurden vom Fachbereich bereitgestellt und durch Deserialisierung in \acs{DTO}-Klassen überführt.\\
Im \Anhang{app:Datenmodell} befindet sich dazu eine Übersicht. 

\subsection{Geschäftslogik}
\label{sec:Geschaeftslogik}
% \begin{itemize}
% 	\item Modellierung und Beschreibung der wichtigsten (!) Bereiche der Geschäftslogik (\zB mit Kom\-po\-nen\-ten-, Klassen-, Sequenz-, Datenflussdiagramm, Programmablaufplan, Struktogramm, \ac{EPK}).
% 	\item Wie wird die erstellte Anwendung in den Arbeitsfluss des Unternehmens integriert?
% \end{itemize}
Unter den Anwendungsfällen ist das Abfragen der Daten von der Datenbank-\acs{API} der wichtigste.
Anhand diesem wird ein Großteil des Programmablaufs nachvollziehbar und kann im Sequenzdiagramm
im \Anhang{app:Sequenzdiagramm} betrachtet werden.\\
Im gezeigten Fall ruft der Nutzer über seinen Browser das \Fachbegriff{ReklaTool} auf.
Dort gibt er über das Suchfeld das Aktenzeichen zum gesuchten Vorgang ein. Da die Aktenzeichen kein festes 
Format besitzen, muss zusätzlich deren Art (Dateiname, Vorgangsnummer, Schadennummer oder Auftragsnummer)
ausgewählt werden. Um die Suche einzuschränken, kann ein Haken gesetzt werden, ob Dokumente zum Vorgang mitgeliefert werden sollen.
Nach Betätigung der Suchen-Taste wird eine Anfrage mit Aktenzeichen und Filter an die 
\Fachbegriff{ReklaTool}-\acs{API} gestellt. Dort bietet der \Klasse{VorgangController} Endpunktmethoden für \acs{HTTP}-Anfragen an.
Über den \Klasse{UiEndpointService} werden die Anfragen an den \Klasse{HttpMsgService} durchgereicht.
Dieser erstellt mittels des \Klasse{RequestBuilder} ein Anfrage-Objekt.\\
Die erzeugte Anfrage wird zunächst durch den \Klasse{ResponseCacheService} mit den bereits zwischengespeicherten
Vorgängen verglichen. Ist eine Datei zum Aktenzeichen vorhanden, so wird diese an den \Klasse{UiEndpointService} 
zurückgegeben. Andernfalls wird nun eine Anfrage an die Datenbank-\acs{API} gestellt.
Das zurückgelieferte \acs{XML}-Dokument wird deserialisiert und zusammen mit den Anfrageinformationen im \Fachbegriff{Cache} abgelegt.
Danach geht auch dieses zurück an den \Klasse{UiEndpointService}.\\
Hier werden die Daten, je nach Client-Anfrage, vom kompletten Vorgangs-Model auf einzelne \Fachbegriff{View-Models} gemappt.
Zurück im \Klasse{VorgangController} wird nun die Server-Response in Form einer \acs{JSON}-Datei zurück an den Client gesendet und die Daten
an die entsprechenden Komponenten gebunden, wo sie für den Nutzer sichtbar werden.

\subsection{Maßnahmen zur Qualitätssicherung}
\label{sec:Qualitaetssicherung}
% \begin{itemize}
% 	\item Welche Maßnahmen werden ergriffen, um die Qualität des Projektergebnisses (siehe Kapitel~\ref{sec:Qualitaetsanforderungen}: \nameref{sec:Qualitaetsanforderungen}) zu sichern (\zB automatische Tests, Anwendertests)?
% 	\item \Ggfs Definition von Testfällen und deren Durchführung (durch Programme/Benutzer).
% \end{itemize}
Um die einzelnen Schritte während der Implementierung nachvollziehbar zu halten,
wird das Projekt in der Quellcodeverwaltung \Fachbegriff{Git} abgelegt.
Zur Vereinfachten Handhabung von \Fachbegriff{Git} wird im Unternehmen auf die Software \Fachbegriff{SmartGit} gesetzt. 
Diese zeigt Änderungen im Projekt an und beherrscht alle typischen Funktionen und Workflows im Zusammenhang
mit der Versionskontrolle. Bei Fertigstellung einer Implementierungseinheit wird das lokale Projekt in das 
Projekt-Repository im Firmennetz gepusht. Zur Verwaltung der Projekte setzt die \acs{Icam} hierbei auf \Fachbegriff{GitLab} 
als on-premises Installation. Zum Zweck des Code-Review werden Mitarbeiter der IT-Abteilung zum \Fachbegriff{GitLab}-Projekt hinzugefügt.\\
Die Qualitätssicherung während der Entwicklung der Webanwendung wurde ursprünglich durch Unit-Tests geplant
(siehe Projektantrag).
Da die einzelnen Services nur auf Basisfunktionen des Frameworks zurückgreifen und die Komplexität des Programms 
überwiegend durch die Verbindung der Module untereinander entsteht, hat sich der Autor auf die Durchführung von
Integrationstests beschränkt. Diese werden in Form von Whitebox-Tests umgesetzt.

\subsection{Pflichtenheft}
\label{sec:Pflichtenheft}
% \begin{itemize}
% 	\item Auszüge aus dem Pflichtenheft/Datenverarbeitungskonzept, wenn es im Rahmen des Projekts erstellt wurde.
% \end{itemize}
% \paragraph{Beispiel}
% Ein Beispiel für das auf dem Lastenheft (siehe Kapitel~\ref{sec:Lastenheft}: \nameref{sec:Lastenheft}) aufbauende Pflichtenheft ist im \Anhang{app:Pflichtenheft} zu finden.
Aus dem Lastenheft (\Anhang{app:Lastenheft}) heraus wurde das Pflichtenheft erstellt. 
Dabei hat der Autor wichtige Implementierungsdetails ergänzt, welche ebenfalls mit der Fachabteilung abgestimmt wurden.
Ein Auszug ist im \Anhang{app:Pflichtenheft} einsehbar. Es dient zur Kontrolle, ob alle Anforderungen planmäßig
umgesetzt wurden.
\clearpage