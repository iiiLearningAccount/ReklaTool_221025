% !TEX root = ../Projektdokumentation.tex
\section{Fazit} 
\label{sec:Fazit}
Durch die Einführung des ReklaTool konnten die Mitarbeiter spürbar entlastet werden.
Die Verkürzung der Abfragedauer sorgt dafür, dass die täglichen Arbeitsabläufe weniger
stark unterbrochen werden. In der gewonnenen Zeit können sich die Nutzer, zu denen auch
Teamleiter gehören, anderen Aufgaben zuwenden. Sobald weitere Nutzer durch den vereinfachten
Prozess hinzugefügt werden, wird sich dieser Effekt noch mehr verstärken. Nicht zu klein zu bewerten
ist der finanzielle Aspekt dieser Optimierung, welche dadurch als Erfolg gewertet werden darf. 

\subsection{Soll-/Ist-Vergleich}
\label{sec:SollIstVergleich}
% \begin{itemize}
% 	\item Wurde das Projektziel erreicht und wenn nein, warum nicht?
% 	\item Ist der Auftraggeber mit dem Projektergebnis zufrieden und wenn nein, warum nicht?
% 	\item Wurde die Projektplanung (Zeit, Kosten, Personal, Sachmittel) eingehalten oder haben sich Abweichungen ergeben und wenn ja, warum?
% 	\item Hinweis: Die Projektplanung muss nicht strikt eingehalten werden. Vielmehr sind Abweichungen sogar als normal anzusehen. Sie müssen nur vernünftig begründet werden (\zB durch Änderungen an den Anforderungen, unter-/überschätzter Aufwand).
% \end{itemize}
% \paragraph{Beispiel (verkürzt)}
In Tabelle~\ref{tab:Vergleich} ist zu erkennen, dass die Zeitplanung überwiegend eingehalten werden konnte.
Über das gesamte Projekt hinweg kam es in zwei Phasen zu Abweichungen von je einer Stunde.\\
\tabelle{Soll-/Ist-Vergleich}{tab:Vergleich}{Zeitnachher.tex}

Durch die gute Kommunikation mit dem Fachbereich bei der Analyse, Planung und Durchführung, konnten die angesetzten Zeiträume  
überwiegend eingehalten werden. Lediglich in der Entwurfsphase wurde eine Stunde mehr aufgewendet. Dieses kam der Implementierungsphase
zugute, welche um eine Stunde verkürzt werden konnte. 

\subsection{Lessons Learned}
\label{sec:LessonsLearned}
% \begin{itemize}
% 	\item Was hat der Prüfling bei der Durchführung des Projekts gelernt (\zB Zeitplanung, Vorteile der eingesetzten Frameworks, Änderungen der Anforderungen)?
% \end{itemize}
Die Umsetzung des Projekts brachte dem Autor viele Erfahrungen im Bereich der Projektplanung und -durchführung.
Dabei nahm die Kommunikation mit den Kollegen der Fachabteilung und der IT-Abteilung eine besondere Stelle ein.
Die Planung und Implementierung der Software waren dazu geeignet, sich weiter in die einzelnen Programmarchitekturen zu vertiefen.
Dabei konnten wichtige Erkenntnisse sowohl über die verwendeten Frameworks, als auch Programmiersprachen gesammelt werden.
Hervorzuheben sind hier die Wirkmechanismen der Kommunikation über \acs{HTTP}.

\subsection{Ausblick}
\label{sec:Ausblick}
% \begin{itemize}
% 	\item Wie wird sich das Projekt in Zukunft weiterentwickeln (\zB geplante Erweiterungen)?
% \end{itemize}
Mit dem Projektabschluss wurden die Anforderungen der Fachabteilung erfüllt.
Durch Veränderung der Prozesse innerhalb des Unternehmens entsteht jedoch durchaus
eine Perspektive auf zukünftige Anpassungen des ReklaTool, um diesem Umstand bestmöglich
Rechnung zu tragen. Optimierungen im Hinblick auf die Erweiterung des Nutzerkreises
sind genauso denkbar, wie die Erweiterung der Darstellung der vom Datenbankservice bereitgestellten Daten.