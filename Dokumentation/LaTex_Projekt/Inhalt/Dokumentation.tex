% !TEX root = ../Projektdokumentation.tex
\section{Dokumentation}
\label{sec:Dokumentation}
% \begin{itemize}
% 	\item Wie wurde die Anwendung für die Benutzer/Administratoren/Entwickler dokumentiert (\zB Benutzerhandbuch, \acs{API}-Dokumentation)?
% 	\item Hinweis: Je nach Zielgruppe gelten bestimmte Anforderungen für die Dokumentation (\zB keine IT-Fachbegriffe in einer Anwenderdokumentation verwenden, aber auf jeden Fall in einer Dokumentation für den IT-Bereich).
% \end{itemize}
% \paragraph{Beispiel}
% Ein Ausschnitt aus der erstellten Benutzerdokumentation befindet sich im \Anhang{app:BenutzerDoku}.
% Die Entwicklerdokumentation wurde mittels PHPDoc\footnote{Vgl. \cite{phpDoc}} automatisch generiert. Ein beispielhafter Auszug aus der Dokumentation einer Klasse findet sich im \Anhang{app:Doc}. 
Um den Einstieg in das Programm zu vereinfachen und dessen Aufbau zu beschreiben,
wurde eine Dokumentation angefertigt. Diese besteht aus dem Benutzerhandbuch für
die Endanwender und der Entwicklerdokumentation. Im Handbuch werden die einzelnen
Funktionen der Webanwendung und die Gliederung der Benutzeroberfläche mit Texten
und Bildern dargestellt. Dadurch wird die Anlernphase verkürzt.\\
Ein Auszug aus dem Benutzerhandbuch befindet sich im \Anhang{app:BenutzerDoku}.

Die Entwicklerdokumentation wurde über Screenshots aus der \acs{IDE} erstellt.
Diese wurden nummeriert und in Textform beschrieben. Ausschnitte aus diesem
Dokument sind im \Anhang{app:EntwicklerDoku} zu finden.

