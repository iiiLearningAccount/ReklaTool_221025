% !TEX root = ../Projektdokumentation.tex
\section{Analysephase} 
\label{sec:Analysephase}
Um die Anforderungen an das ReklaTool zu bestimmen wurde zusammen mit der Fachabteilung der
aktuelle Prozess zur Reklamationsbearbeitung erfasst. Aus der Analyse des Vorgangs ergaben sich
einzelne Anwendungsfälle (Use-Cases) für die zu erstellende Webanwendung.
Basierend auf dem Umfang der Anwendung wurde eine Wirtschaftlichkeitsanalyse durchgeführt und
schließlich das Lastenheft erstellt.

\subsection{Ist-Analyse} 
\label{sec:IstAnalyse}
% \begin{itemize}
% 	\item Wie ist die bisherige Situation (\zB bestehende Programme, Wünsche der Mitarbeiter)?
% 	\item Was gilt es zu erstellen/verbessern?
% \end{itemize}
Um den im Folgenden beschriebenen Prozess zu verbildlichen, wurde das im \Anhang{app:Aktivitaet} befindliche 
Schaubild erstellt.\\
Zum Leistungsspektrum der IcamSystems gehört die automatisierte Prüfung von Gutachten und
Kostenvoranschlägen, welche durch Gutachter oder Partnerwerkstätten erstellt wurden.
Die Prüfung wird durch die Versicherungen oder Sachverständigenorganisationen (Kunden) veranlasst
und wird durch den \acs{CG} durchgeführt.
Als Ergebnis bekommt der Kunde eine Rückmeldung über die ausgelösten Regeln des \acs{CG}.\\
Im Fall, dass die Regelauslösungen für den Kunden nicht nachvollziehbar sind, hat er die
Möglichkeit eine Reklamation an die IcamSystems GmbH zu senden. Diese wird firmenintern geprüft
und an die Fachabteilung Prozessautomatisierung weitergeleitet.\\
Ein/-e Mitarbeiter/-in des Teams Reklamationsbearbeitung recherchiert anschließend mittels des Aktenzeichens
des Vorgangs die dazugehörigen Daten. Je nach Zeitpunkt des Vorgangs können die Daten in verschiedenen Datenbanken 
(Produktiv-, Archiv- und Testdatenbank) abgelegt sein. Der/die Mitarbeiter/-in sucht mittels \ac{DBMS} und selbst 
erstellten SQL-Skripten nach dem Vorgang.\\
Mit diesen Daten kann dann die beanstandete Regelauslösung nachvollzogen werden. Dazu bietet der \acs{CG} die 
Möglichkeit Vorgänge Schritt für Schritt durchzugehen. Fällt dabei eine fehlerhafte Prüfregel auf, so wird diese ergänzt 
oder angepasst.\\
In einem weiteren Schritt wird die Reklamation an die Rechercheabteilung gegeben. Diese prüft die Fahrzeugdaten des
Vorgangs auf Vollständigkeit und Korrektheit. Auch hier werden bei Auffälligkeiten Daten korrigiert oder nachrecherchiert
und ergänzt.\\
Der Kunde bekommt Rückmeldung über die angepassten Daten und Prüfregeln. 
Sind die Daten jedoch nach Prüfung initial richtig gewesen, so bekommt der Kunde auch darüber eine Rückmeldung, 
inklusive von Quellen (\zB Herstellerdaten) als Nachweis.
Danach ist der Prozess abgeschlossen und kann für den nächsten Vorgang von vorne beginnen.


\subsection{Wirtschaftlichkeitsanalyse}
\label{sec:Wirtschaftlichkeitsanalyse}
% \begin{itemize}
% 	\item Lohnt sich das Projekt für das Unternehmen?
% \end{itemize}
Wie bereits im Abschnitt \nameref{sec:IstAnalyse} zu erkennen ist, war der vorherige Reklamationsprozess
mit viel manueller Recherchearbeit verbunden. Im Folgenden wird analysiert, ob das Projekt durch die Zeitersparnis,
welche mit dem ReklaTool einhergeht, wirtschaftlich sinnvoll ist. 


\subsubsection{\gqq{Make or Buy}-Entscheidung}
\label{sec:MakeOrBuyEntscheidung}
% \begin{itemize}
% 	\item Gibt es vielleicht schon ein fertiges Produkt, dass alle Anforderungen des Projekts abdeckt?
% 	\item Wenn ja, wieso wird das Projekt trotzdem umgesetzt?
% \end{itemize}
Das ReklaTool greift mittels des Webservice auf sensible Firmendaten zu, mit denen auch Verpflichtungen gegenüber
Kunden einhergehen. Weiterhin soll die Webanwendung unternehmensspezifischen Anforderungen genügen.
Aus diesen Gründen ist von dem Beziehen von Software von Dritten abzusehen und die Eigenentwicklung vorzuziehen.


\subsubsection{Projektkosten}
\label{sec:Projektkosten}
% \begin{itemize}
% 	\item Welche Kosten fallen bei der Umsetzung des Projekts im Detail an (\zB Entwicklung, Einführung/Schulung, Wartung)?
% \end{itemize}

Dieser Abschnitt betrachtet die Kosten, die für die Umsetzung des Projekts entstehen.\\ 
Diese setzen sich sowohl aus Personalkosten, als auch aus Kosten für verwendete Ressourcen
(siehe Kapitel~\ref{sec:Ressourcenplanung}: \nameref{sec:Ressourcenplanung}) zusammen.\\
Die brutto Personalkosten je Projektmitarbeiter wurden durch die Personalabteilung vorgegeben. 
Für einen Mitarbeiter der IT-Abteilung wurde ein Stundensatz in Höhe von \eur{50,00} angenommen. Mitarbeiter
der Fachabteilung gehen mit \eur{70,00} pro Stunde in die Berechnung ein. Da der Autor aufgrund seiner Umschulung
nicht direkt von der Praktikumsfirma bezahlt wird, wird für diesen das Azubigehalt des dritten Lehrjahrs,
mit \eur{10,00} pro Stunde angenommen.\\
Die Betriebskosten für die Webanwendung werden mit jährlich \eur{100,00} beziffert und für die Nutzung der Ressourcen 
gilt ein Pauschalbetrag von \eur{20,00}. \\
Die Gesamtkosten des Projekts betragen somit \eur{}. Die genaue Berechnung kann der Tabelle~\ref{tab:Kostenaufstellung}
entnommen werden.

% \paragraph{Beispielrechnung (verkürzt)}
% Die Kosten für die Durchführung des Projekts setzen sich sowohl aus Personal-, als auch aus Ressourcenkosten zusammen.
% Laut Tarifvertrag verdient ein Auszubildender im dritten Lehrjahr pro Monat \eur{1000} Brutto. 

% \begin{eqnarray}
% 8 \mbox{ h/Tag} \cdot 220 \mbox{ Tage/Jahr} = 1760 \mbox{ h/Jahr}\\
% \eur{1000}\mbox{/Monat} \cdot 13,3 \mbox{ Monate/Jahr} = \eur{13300} \mbox{/Jahr}\\
% \frac{\eur{13300} \mbox{/Jahr}}{1760 \mbox{ h/Jahr}} \approx \eur{7,56}\mbox{/h}
% \end{eqnarray}

% Es ergibt sich also ein Stundenlohn von \eur{7,56}. 
% Die Durchführungszeit des Projekts beträgt 70 Stunden. Für die Nutzung von Ressourcen\footnote{Räumlichkeiten, Arbeitsplatzrechner etc.} wird 
% ein pauschaler Stundensatz von \eur{15} angenommen. Für die anderen Mitarbeiter wird pauschal ein Stundenlohn von \eur{25} angenommen. 
Eine Aufstellung der Kosten befindet sich in Tabelle~\ref{tab:Kostenaufstellung} und sie betragen insgesamt \eur{2739,20}.
\tabelle{Kostenaufstellung}{tab:Kostenaufstellung}{Kostenaufstellung.tex}


\subsubsection{Amortisationsdauer}
\label{sec:Amortisationsdauer}
% \begin{itemize}
% 	\item Welche monetären Vorteile bietet das Projekt (\zB Einsparung von Lizenzkosten, Arbeitszeitersparnis, bessere Usability, Korrektheit)?
% 	\item Wann hat sich das Projekt amortisiert?
% \end{itemize}
Der Wegfall der in Abschnitt \ref{sec:IstAnalyse}~\nameref{sec:IstAnalyse} beschriebenen Recherchearbeit in mehreren Datenbanken schlägt sich
in einer Zeitersparnis nieder, welche sich \ua über die Lohnkosten als finanzieller Vorteil beziffern lässt.\\
Die Ermittlung der Amortisationsdauer soll dabei helfen, die Wirtschaftlichkeit des Projekts zu beurteilen.
Für die Berechnung wird die vorher im Abschnitt  \ref{sec:Projektkosten}~\nameref{sec:Projektkosten} kalkulierte Gesamtsumme mit den Einsparungen
verglichen. An der Stelle, wo sich Projektkosten und Einsparung nach einer bestimmten Zeit treffen, lässt sich der
Break-Even Point ablesen. Durch diesen hat man eine Aussage darüber, ab welchem Zeitpunkt sich das Projekt amortisiert hat.

% \paragraph{Beispielrechnung (verkürzt)}
% Bei einer Zeiteinsparung von 10 Minuten am Tag für jeden der 25 Anwender und 220 Arbeitstagen im Jahr ergibt sich eine gesamte Zeiteinsparung von 
% \begin{eqnarray}
% 	25 \cdot 220 \mbox{ Tage/Jahr} \cdot 10 \mbox{ min/Tag} = 55000 \mbox{ min/Jahr} \approx 917 \mbox{ h/Jahr} 
% \end{eqnarray}
% Dadurch ergibt sich eine jährliche Einsparung von 
% \begin{eqnarray}
% 	917 \mbox{h} \cdot \eur{(25 + 15)}{\mbox{/h}} = \eur{36680}
% \end{eqnarray}
% Die Amortisationszeit beträgt also $\frac{\eur{2739,20}}{\eur{36680}\mbox{/Jahr}} \approx 0,07 \mbox{ Jahre} \approx 4 \mbox{ Wochen}$.


% \subsection{Nutzwertanalyse}
% \label{sec:Nutzwertanalyse}
% \begin{itemize}
% 	\item Darstellung des nicht-monetären Nutzens (\zB Vorher-/Nachher-Vergleich anhand eines Wirtschaftlichkeitskoeffizienten). 
% \end{itemize}
% \paragraph{Beispiel}
% Ein Beispiel für eine Entscheidungsmatrix findet sich in Kapitel~\ref{sec:Architekturdesign}: \nameref{sec:Architekturdesign}.
\subsection{Nicht-monetärer Nutzen}
\label{sec:nichtMonetaererNutzen}
Der alte Reklamationsprozess (\Vgl Kapitel \ref{sec:IstAnalyse} \nameref{sec:IstAnalyse}) wurde aufgrund seiner Komplexität
meist von \ac{SME} durchgeführt. Durch die hohe Priorität der Reklamationen wurde der Arbeitsablauf dieser Mitarbeiter für 
einen längeren Zeitraum unterbrochen. Diese Unterbrechung wurde durch den automatisierten Zugriff auf alle relevanten Quellen mit 
dem ReklaTool verkürzt. Durch die einfachere Handhabung ist nun auch die Einbeziehung weiterer Mitarbeiter in das 
Reklamationsmanagement möglich.\\
Weiterhin wird durch die Vereinheitlichung des Prozesses und die Reduzierung manueller Teilschritte, 
die Fehlerquote minimiert werden. 

\subsection{Anwendungsfälle}
\label{sec:Anwendungsfaelle}
% \begin{itemize}
% 	\item Welche Anwendungsfälle soll das Projekt abdecken?
% 	\item Einer oder mehrere interessante (!) Anwendungsfälle könnten exemplarisch durch ein Aktivitätsdiagramm oder eine \ac{EPK} detailliert beschrieben werden. 
% \end{itemize}
% \paragraph{Beispiel}
% Ein Beispiel für ein Use Case-Diagramm findet sich im \Anhang{app:UseCase}.
Zusammen mit der Fachabteilung wurden während der Analyse Anforderungen definiert und daraus Anwendungsfälle
abgeleitet. Im \Anhang{app:UseCase} befindet sich das dabei entstandene Use-Case-Diagramm.\\
Die Ausdifferenzierung der einzelnen Fälle diente dann später im Entwurf als Richtlinie für einzelne Features.

\subsection{Qualitätsanforderungen}
\label{sec:Qualitaetsanforderungen}
% \begin{itemize}
% 	\item Welche Qualitätsanforderungen werden an die Anwendung gestellt (\zB hinsichtlich Performance, Usability, Effizienz \etc (siehe \citet{ISO9126}))?
% \end{itemize}
Die Software soll eine funktionale und übersichtliche Benutzeroberfläche bieten.
Diese soll ohne Installation, in Form einer Webanwendung im Browser erreichbar sein.
Nach einer Nutzeranfrage an die Anwendung soll das Ergebnis in einer festgelegten Zeit erscheinen,
um den Arbeitsablauf nicht zu sehr zu stören.
Die Funktionalität der Software soll mittels Tests sichergestellt werden.


\subsection{Lastenheft/Fachkonzept}
\label{sec:Lastenheft}
% \begin{itemize}
% 	\item Auszüge aus dem Lastenheft/Fachkonzept, wenn es im Rahmen des Projekts erstellt wurde.
% 	\item Mögliche Inhalte: Funktionen des Programms (Muss/Soll/Wunsch), User Stories, Benutzerrollen
% \end{itemize}

% \paragraph{Beispiel}
% Ein Beispiel für ein Lastenheft findet sich im \Anhang{app:Lastenheft}. 
Aus den zusammengetragenen Anforderungen der Fachabteilung wurde gemeinsam mit dieser
das Lastenheft erstellt. Darin sind alle vorgegebenen Funktionen und Eigenschaften der zu erstellenden 
Software verbindlich festgeschrieben.\\
Ein Auszug befindet sich im \Anhang{app:Lastenheft}.