% !TEX root = ../Projektdokumentation.tex
\section{Projektplanung} 
\label{sec:Projektplanung}

\subsection{Projektphasen}
\label{sec:Projektphasen}
%\begin{itemize}
%	\item In welchem Zeitraum und unter welchen Rahmenbedingungen (\zB Tagesarbeitszeit) findet das Projekt statt?
%	\item Verfeinerung der Zeitplanung, die bereits im Projektantrag vorgestellt wurde.
%\end{itemize}
Für die Bearbeitung des Projekts standen dem Autor im Projektzeitraum täglich etwa 5 Stunden zur Verfügung.
Insgesamt wurde das Projekt in 80 Stunden umgesetzt. Diese Zeit wurde in Phasen aufgeteilt, welche den
Projektablauf widerspiegeln. Aus Tabelle~\ref{tab:Zeitplanung} kann die grobe Zeitplanung entnommen werden.
\tabelle{Zeitplanung}{tab:Zeitplanung}{ZeitplanungKurz}\\
Eine detailliertere Zeitplanung befindet sich im \Anhang{app:Zeitplanung}.

% \subsection{Abweichungen vom Projektantrag}
% \label{sec:AbweichungenProjektantrag}
% \begin{itemize}
% 	\item Sollte es Abweichungen zum Projektantrag geben (\zB Zeitplanung, Inhalt des Projekts, neue Anforderungen), müssen diese explizit aufgeführt und begründet werden.
% \end{itemize}

\subsection{Ressourcenplanung}
\label{sec:Ressourcenplanung}
% \begin{itemize}
% 	\item Detaillierte Planung der benötigten Ressourcen (Hard-/Software, Räumlichkeiten \usw).
% 	\item \Ggfs sind auch personelle Ressourcen einzuplanen (\zB unterstützende Mitarbeiter).
% 	\item Hinweis: Häufig werden hier Ressourcen vergessen, die als selbstverständlich angesehen werden (\zB PC, Büro). 
% \end{itemize}
Im \Anhang{app:Ressourcen} befindet sich eine Auflistung der verwendeten Ressourcen.
Dort werden sowohl Hardware- und Softwareressourcen, als auch das beteiligte Personal aufgeführt.
Es wurde nur auf Software zurückgegriffen, für die bereits Lizenzen im Unternehmen vorhanden war,
\acs{bzw} die kostenfrei genutzt werden konnte. Dieser Umstand wirkte sich positiv auf die Projektkosten aus.

\subsection{Entwicklungsprozess}
\label{sec:Entwicklungsprozess}
%\begin{itemize}
%	\item Welcher Entwicklungsprozess wird bei der Bearbeitung des Projekts verfolgt (\zB Wasserfall, agiler Prozess)?
%\end{itemize}
Das Projekt wird vom Autor als einzelner Entwickler in einem überschaubaren Zeitraum 
von drei Wochen umgesetzt. Aus diesen Gründen und da die Anforderungen zu Beginn schon
klar definiert wurden, wurde das Projekt anhand der Phasen des Wasserfallmodells umgesetzt. 
Eine Eigenschaft des Wasserfallmodells ist die lineare Abfolge der einzelnen Projektphasen.
Bei der Implementierung wurde bewusst ein inkrementeller Ansatz gewählt, um die Produktqualität
zu gewährleisten.\\
Besonders bei Analyse und Entwurf ist die Rücksprache mit dem Fachbereich
vorgesehen.\\
Zur Sicherstellung der späteren Wartbarkeit wurde die Webanwendung nach den fünf SOLID-Prinzipien
für objektorientierte Programmierung entworfen.\\ 
Um die einzelnen Schritte während der Implementierung nachvollziehbar zu halten,
wurde das Projekt in die Versionsverwaltung für Quellcode via \Fachbegriff{Git} integriert.
Dazu wurden Änderungen im lokalen Projektverzeichnis mit dem Clientprogramm SmartGit überwacht.
Bei Fertigstellung einer Implementierungseinheit wurde das lokale Projekt in das 
Remote-Repository im Firmennetz gepusht. Die Versionsverwaltung in der Firma erfolgt über eine lokal gehostete Instanz
von GitLab. Zum Zweck des Codereview wurden Mitarbeiter der IT-Abteilung zum GitLab-Projekt hinzugefügt.\\
Die Qualitätssicherung während der Entwicklung der Webanwendung wurde durch Unit-Tests umgesetzt. Die Tests 
sorgen dafür, dass das erwartete Verhalten einzelner Komponenten sichergestellt wird. Weiterhin ist dadurch 
sichergestellt, dass zukünftige Änderungen an der Anwendung deren Lauffähigkeit nicht beeinträchtigen.\\
Die Integration der einzelnen Module der Anwendung wurde während der Entwicklung immer wieder durch Whitebox-Tests
sichergestellt.